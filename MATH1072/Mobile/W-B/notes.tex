\documentclass[a6paper]{article}
\usepackage[margin=5mm]{geometry}

\usepackage{xcolor}
\usepackage{amsmath}
\usepackage{amsthm}
\usepackage{amssymb}

\theoremstyle{definition}
\newtheorem{definition}{Definition}
\newtheorem{example}{Example}

\theoremstyle{plain}
\newtheorem{theorem}{Theorem}
\newtheorem{lemma}{Lemma}
\newtheorem{cor}{Corollary}
\newtheorem{prop}{Proposition}
\newtheorem{conj}{Conjecture}

\theoremstyle{remark}
\newtheorem*{remark}{Remark}
\newtheorem{note}{Note}
\newtheorem{case}{Case}


\title{MATH1072 Notes}
\author{Ismael Khan}

\definecolor{bluegray}{RGB}{147, 148, 186}

%\pagecolor{black}
%\color{white}

\usepackage{listings}
\usepackage{color}

\definecolor{dkgreen}{rgb}{0,0.6,0}
\definecolor{gray}{rgb}{0.5,0.5,0.5}
\definecolor{mauve}{rgb}{0.58,0,0.82}

\lstset{frame=tb,
  language=Java,
  aboveskip=3mm,
  belowskip=3mm,
  showstringspaces=false,
  columns=flexible,
  basicstyle={\small\ttfamily},
  numbers=none,
  numberstyle=\tiny\color{gray},
  keywordstyle=\color{blue},
  commentstyle=\color{dkgreen},
  stringstyle=\color{mauve},
  breaklines=true,
  breakatwhitespace=true,
  tabsize=3
}

\begin{document}
	\maketitle
	\section{Lecture 1 - Introduction to Dimensional Analysis}
	To descrive real systems quantitatively, we use numbers and units of
	measurement. Eg. 3 meters, 5 years, 10 $ km/h $.
	Each measurable quantity has a certain dimension.
	\subsection{Base dimensions}
	Length $ (L) $, time $ (T) $, mass $ (M) $, other non-mechanical base
	dimensions include temperature; electric charge/current etc.
	\subsection{Derived dimensions}
	Speed $ \displaystyle \frac{L}{T} $, force $ (
	N = M \displaystyle \frac{L}{T^2}) $, energy $ (J = M \displaystyle
	\frac{L^2}{T^2}) $. The dimensions of the terms added on both sides
	must be equal. This is known as the equation being
	\textit{dimensionally homogenous}. We can use the dimensional
	homogeneity to make dimensional estimateas for certain quantities
	(order of magnitude, not exact prediction)
	\section{Lecture 2 - Dimensional Analysis}
	\section{Lecture 3 - Introduction to Differential Equations}
	\noindent Generally, an ordinary differential equation (ODE) is represented as:
	$$ F(t,y(t), y'(t), y''(t), ...)  = 0 $$
	For Instance, Newton's Law
	$$ m \frac{d^2 r}{dt^2} = F $$
	Induction Law:
	$$ RI + L \frac{dI}{dt} + \frac{1}{c} $$
	Population:
	$$ \frac{dP}{dt} = rP(1- \frac{P}{k})$$
	Maxwell's Equations:
	\begin{align*}
		\nabla \cdot \bar{E} &= \frac{1}{\rho} \varepsilon_0 \\
		\nabla \cdot \bar{B} &= 0 \\
		\nabla \times \bar{E} &= \frac{\delta B}{\delta t} \\
		\nabla \times \bar{B} &= \mu_0 J + \mu_0 \varepsilon_0 \frac{\delta E}{\delta t}
	\end{align*}

	Navier-Stokes: (Modelling velocity of fluids in space)
	$$ \frac{\delta \bar{v}}{\delta t} = \overline{v} \nabla \overline{v} = ? $$

	Schrodinger Wave Equation:
	$$ i\hbar \frac{\delta \psi}{\delta t} = \Big[ \frac{-\hbar}{2m} \nabla^2 + V(r)\Big]\psi$$
	$$ \psi = ? $$
	We generally have:
	$$ F(t, y(t), y'(t), ...) = 0,\; y(t) = ? $$
	Where the order of ODE $ = $ order of the higest derivative. Take the equation $y' = y$, the solutions are $ y = e^t $, $ y = 0 $ and $ y = ce^t $. However the latter expression encapsulates the former, so $ y(t) = ce^t $ is known as the general solution where $ c \in \mathbb{R} $. The genreal solution is not unique. \\\\
	If you take an ODE and add some initial condition, then the solution is a unique result. For Instance take the ODE $ y' = y $ and say that $ y(0) = 1 $, then we achieve a unique solution of $ c = 1 $, $ y = e^t $.
      \section{Lecture 4 - Ordinary Differential Equations}
      \subsection{Equilibrium Solutions}
      An equilibrium solution (steady state solution) of an ODE (if such
      solution exists) is a constant solution $ y(t) = c $ which satisfies the
      ODE (for any $ t $). 
      \begin{example}
	Take $ y' = f(t,y) $, the equilibrium solution 
	$$ f(t,y = c) = 0 $$
      \end{example}
      \noindent This implies that the slope at $ y = c $ must be 0 for
      a function defined on a $ (t,y) $ plane

      \begin{example}
      $ y' = y $, $ y = 0 $ is an equilibrium solution.
      \end{example}

      \begin{example}
	$ y' = y(1-y) $m $ y = 0 $ and $ y = 1 $ are equilibrium solutions
      \end{example}
      \noindent Slope fields can visualised with Mathematica using SteamPlot
      (or VectorPlot). The vector flow of a field corresponding to $ f(t,y)
      $ is $ \{1, f(t,y)\} $ since the flow in the horizontal direction $ (t)
      $ has a constant rate (the flow of time) which can be set to $ 1 $, the
      vertical flow is $ f(t,y) $.

      \subsection{Stability of equilibrium solutions}
      If the solutions starting in condition a small neighbourhood of an
      equilibrium soltion $ (y = c) $ converge towards the equilibrium solution
      for large $ t $, then the equilibrium solution $ y = c $ is stable.

      \begin{example}
	$ y' = y $, the equilibrium solution $ y = 0 $ is unstable (for $ y'
	= -y $, $ y = 0 $ is stable)
      \end{example}

      \begin{example}
	$ y' = y(1-y) $, $ y = 0 $ is unstable, but $ y = 1 $ is stable.
      \end{example}

%Indenting LaTeX files? Is it a mandatory thing? Spacing can cause actual
      %typsetting errors but Indenting does not...
      % Regardless, Vim decides to indent for me anyways because of the
      % document environment.
      %

\section{Lecture 5 - Stability of ODE's}
\subsection{Condition for stability of equilibrium soltuions}
Using a Taylor Series approximation of $ f(t,y) $ near the equilibrium solution
$ y = c $, assume $ f(t,y) = f(y) $ for simplicity
$$ f(y) \approx f(c) + f'(c)(y-c) + \frac{1}{2} f''(c)(y-c)^2 + ... $$
Where $ y = c  $ is the equilibrium solution $ f(c) = 0 $, thus
$$ y'(t) = 0 + f'(c)(y-c) + \frac{1}{2} f''(c)(y-c) + ... $$
Take $ u = y -c $, then $$ u' = f'(c)u + ... $$
Implying
$$ u(t) = Ae^{f'(c) t} $$
If $ f'(c) > 0 $, $ y = c $ is unstable. If $ f'(c) < 0 $, $ u(t) \to 0 $,
$ y(t) \to c $ thus $ y =c  $ is stable.
\subsection{Euler's Method}
An iterative operation which models $ y_k \approx y(k\cdot \Delta t)$. Given
$ y' = f(t,y) $, $ y(0) = c$ 
\begin{align}
  y'(t) &\equiv \lim_{\Delta \to 0} \frac{y(t+\Delta) - y(t)}{\Delta}\\ &\approx \frac{y(t+\Delta) - y(t)}{\Delta}
\end{align}
As it is an iterative method, $ y_{k+1} = y_{k} + f(t_k, y_k) $
$$ \frac{y(t_{k+1})-y(t_k)}{\Delta} \approx f(t_k, y_k) $$
\begin{example}
  Given $ y' = 2t $, $ y(t) = ? $, $ y(0) = 0 $. By Euler's Method
  \begin{align}
    N &= 1, \; \Delta = 1 \\
    N &= 2, \; \Delta = \frac{1}{2}\\
    N &= 4, \; \Delta = \frac{1}{4}\\
    N &= 10 \; \Delta = \frac{1}{10}
  \end{align}
\end{example}

\section{Lecture 6 - Continuation of Euler's Method}
\subsection{Error Generated in Euler's Method}
Assume $ y(t_k) = y_k $, the error is represented with the taylor series
approximation 
\begin{align*}
  \Big | y_{k+1} - y(t_k + \Delta) \Big | &= y(t_{k+1}) = y(t_k)
+ y'(t_k)\Delta + \frac{1}{2}y''(t_k) \Delta^2 + ... \\
  y_{k+1} &= y_k + f(t_k, y_k)\Delta
\end{align*}
Thus $$ \text{error} = |y_{k+1} - y(t_k + \Delta)| \propto \Delta^2 $$
However generalized for $ N $ steps,
\begin{align*}
  N\cdot|y_{k+1} - y(t_k + \Delta)| &\propto \Delta^2 \cdot N \\
				    &\propto \Delta^2 \cdot \frac{1}{\Delta}
				    \propto \Delta
\end{align*}

$$ y_{k+1} = y_k + \frac{1}{2}\Big[f(t_k, y_k) + f(t_{k+1}, y_{k+1}) \Big]
\Delta$$

\section{Lecture 7 - Solving Linear First Order ODEs}
\[y' = f(t,y), y(0) = y_0\]
\[t \in [0,t_{max}]\]

\begin{enumerate}
      \item Euler's Method
      \[y_{k+1} = y_k + f(t_k, y_k) \Delta\]
      Where total error \( \propto \Delta\)
      \item Heun Method (Revised Euler's Method)
      \[\begin{cases}
            y_{k+1} = y_k + f(t_k, y_k) \Delta \\
            y_{k+1} = y_k + \frac{1}{2} [f(t_k, y_k) + f(t_k + \Delta, y_{k+1})]
      \end{cases}\]
      Where total error \(\propto \Delta^2\). Note that Heun Method will possibly show in
      Assignment 3.
\end{enumerate}
\subsection{Runge Kutta Method}
\begin{align*}
      p_1 &= f(t_k, y_k) \Delta\\
      p_2 &= f(t_k + \frac{\Delta}{2}, y_k + \frac{p_1}{2})\Delta\\
      p_3 &= f(t_k + \frac{\Delta}{2}, y_k + \frac{p_2}{2})\Delta\\
      p_4 &= f(t_k + \frac{\Delta}{2}, y_k + p_3) \Delta
\end{align*}
for \[y_{k+1} = y_k + \frac{1}{6} p_1 + \frac{1}{3} p_2 + \frac{1}{3} p_3 + \frac{1}{6}p_4 + ...\]
\subsection{Adaptive Step Size}
Fixed step size is mostly inefficient in most cases, so we use an adaptive step size
for numerical methods to achieve better approximations.

\subsection{Coupled Systems}
\begin{align*}
      y' &= f(t,y_1, y_2) \\
      x' &= g(t, y_1, y_2) 
\end{align*}
\[x(t) = ? \; y(t) = ?\]
In the context of Euler's Method
\[\begin{cases}
      x_{k+1} = x_k + f(t_k , x_k, y_k)\Delta\\
      y_{k+1} = y_k + g(t_k, x_k, y_k)\Delta
\end{cases}\]

\subsection{Analytical ODE Solutions}
\subsubsection{Linear First Order ODE's}
By definition,
\[y(t) = f(t)y(t) + g(t)\]
is generally the standard form of a linear first order ODE. Note that 
\[y'(t) = f(t)y(t)\]
is a special case of a linear first order ODE that is seperable.

\begin{example}
      \[ty' + y = t\cos t\]
      Observe that, by the product rule for differentiation, \(ty' + y = (ty)'\).
      \begin{align*}
            (ty)' &= t\cos t\\
            \int (ty)' dy &= \int t \cos t dt\\ 
            ty &= t\sin t - \int \sin t dt \\
            ty &= t\sin t + \cos t + c\\
            \therefore y &= \frac{t\sin t + \cos t}{t} + \frac{c}{t}
      \end{align*}

      
\end{example}
% Lectures 9-11 need to be updated to the notes. Refer to blackboard content
% and 

\section{Lecture 8 - Applications of First Order ODEs}
Common applications of first order ODEs are
\subsection{Radioactive Decay}
The particles of a radioactive material decay spontaneously in a stochastic
process. The total mass of the radioactive atoms decrease with time. We can represent this as 
$$ \frac{dM}{dt} = -kM $$
For $ M(t) $ to represent the mass of the radioactive material over time $ t $.
% This is all one line of text, maybe add some line breaks...
\noindent
Clearly the solution to this is 
$$ M(t) = M_0 e^{-kt} $$
Note that there is a limitation to this ODE Model. We assume that the mass changes \underline{continuously} in time. Whereas in reality it 
changes in discrete steps following individual decay events. However 
for the macroscopic mass, the number of particles is much larger, so we can neglect the discrete jumps and assume a continuous deterministic model. Which works well for most cases.
\\ \par
The lifetime of particles varies, however we can categorize them
with their average lifetime. We do this by asking how long it
takes for a particle to reduce to half of the initial value.
To obtain a more accurate value for average lifetime, consider
grouping the lifetime of particles into discrete "bins". If the
number of particles with lifetime in $ [t_j, t_j + \Delta] $
is $ N_j $, then the average lifetimes is represented as
$$ \sum N_j = N_0 $$
$$ \tau \approx \frac{ \displaystyle \sum t_j N_j}{\displaystyle \sum N_j} $$
If the number of particles decreases exponentially
$$ N(t) = N_0 e^{-kt} $$
Then the number of particles lost in an interval $ [t_j, t_j + \Delta] $ is 
$$ N_j = N[t_j] - N[t_j + \Delta] $$
Or
$$ N_j = \frac{N(t_j) - N(t_j + \Delta)}{\Delta} \cdot \Delta
\approx - N'(t)\Delta$$
Taking $ \Delta \to 0 $. The sum for calculating the average lifetime turns into an integral.
$$ \tau = \frac{1}{N_0} \int_0^\infty  t(-N'(t)) \; dt = 
\int_0^\infty te^{-kt} \; dt = \frac{1}{k}$$
\subsection{Protein Synthesis and Degradation}
Seriously who cares. I probably should write these notes though
\section{Lecture 9 - Continuation of Protein Synthesis and Degradation}
The concentration of a protein $ C(t) $ that is synthesised at a constant rate
$ S $, $ k $ for the degradation rate.
$$ \frac{dC}{dt} = S - kC $$
We can see that the equilibrium state is at $ C = \frac{S}{k} $, the ODE is
seperable, so the general solution is as follows
$$ \int \frac{dC}{S-kC} \; = \int dt $$
And then... Poof! By the magic of Applied Mathematics, we achieve the general
solution
$$ C(t) = \frac{S}{k} + ce^{-kt} $$
Assume the initial value such that for $ C(t=0) = C_0 $, then the general
solution
$$ C(t) = \frac{S}{k} + \Big (C_0 - \frac{S}{k} \Big) e^{-kt} $$

% Insert graphic from Lecture Notes???

\section{Lecture 10 - Heat Transfer}
The heat flow (energy transferred per unit time) between two object of
different temperature is proportional to the temperature difference between the
two object (the heat flows from the higher to the low temperatures until it
equilibrates). Thus the rate of change of the body temperature is proportional
to the temperature difference between the body and the environment $ T_{env}
$ (air). 
$$ \frac{dT}{dt} = - k(T-T_{env})  $$
$$ T(t) = T_{env} + (T_0 - T_{env})e^{-kt} $$
Setting $ t = 0 $ solves for $ T_0 $ and $ T_{env} $ should be predefined. The
parameter $ k $ is a proportionality constant and is difficult to represent
with a model (at this level)
\subsection{Electric circuit with resistance and inductance}
$ R $ resistance, $ L $ inductance in series, $ E(t) $ external voltage source
$ = $ the sum of the voltage drops over the resistance + the inductance.
The ODE for the current $ I(t) $,
$$ E(t) = RI + L \frac{dI}{dt} $$
\subsection{Population Dynamics}
The rate of change of the population $ P(t) $ is proportional with the actual
current population 
$$ \frac{dP}{dt} = kP $$, $$ P(t) = P_0 e^{kt} $$
The $ k $ parameter in this model is the cell division rate constant. Ideally,
bacteria can divide every 20 minutes.
\section{Lecture 11 - Continuation of Population Dynamics}
To balance out the growth of population due to reproduction we can also include
a loss term due to death. The number of individuals dying per unit time is also
proportional to population size $ P $
$$ \frac{dP}{dt} = kp - dP = (k-d)P = \rho P$$
$$ P(t) = P_0 e^{\rho t} $$
This modified population model still doesn't have a stable equilibrium state.
The solution grows or decays exponentially depending on the sign of $ \rho
= k - d $ which is the \textbf{net reproduction rate constant}.
It is expected that a population should stabilise after some time in a stable
equilibrium; where repoduction and death balance out.
\par The thing missing from this model is that we assumed $ k $ and $ d $ are
constant parameters, but the birth and death rates may be dependant on several
external factors (e.g. food, habitat) and this may be dependent on the size of
the population $ r $, $ k $ (or $ \rho $) are functions of $ P $. 
\\\\
Thus a modified nonlinear population model
$$ \frac{dP}{dt} = \rho (P) P $$
Where the exact form of the function $ \rho (P) $ depends on the problem in
question that we want to model. In general $ \rho (P) $ is a decreasing
function (less resources available when the population increases, which slows
down reproduction and/or increases death rate).
\\\\
The simplest functional form for a decreasing $ \rho (P) $ is a linear function
$$ \frac{dP}{dt} = \rho \Big (1 - \frac{P}{K} \Big ) P $$
This is known as the \textbf{Logistic Equation}, where $ \rho  $ is a constant
(maximum net reproduction rate), $ K $ is the carrying capacity (the maximum
sustainable population size). The net reproduction changes sign from positive
to negative when $ P = K $. The equilibrium solutions of this model are 
\begin{itemize}
  \item $ P^* = 0 $; unstable
  \item $ P^* = K $; the derivative of RHS at $ P = K $ is $ < 0 $ implying
    stable equilibrium state.
\end{itemize}
The exact solution to this ODE (seperable ODE) with initial condition $ P(0)
= P_0 $ is 
$$ P(t) = K \frac{1}{1+\Big( \frac{K}{P_0} - 1\Big) e^{-\rho t}} $$
For a long time $ t \to \infty $, the solution $ P(t) $ converges to the stable
equilibrium. Assume that $ P $ describes a fish population and there is an
additional loss rate due to fishing.
$$ \frac{dP}{dt} = \rho \Big (1 - \frac{P}{K}\Big ) P - f $$
$ f $ is a constant parameter; the amount of fish harvested per unit time.
\subsubsection*{Question:}
How does the fishing rate $ f $ modify the state equilibrium of the population?
How should the function $ P^* (f) $ look graphed?. Is there any qualitative
(dramatic) change of the equilibrium as $ f $ is modified or only a smooth
transition; are there any bifuractions?
\section{Lecture 12 - ODE Models of Population Dynamics}
\subsection{Bifuraction}
Consider the Differential equation $ \displaystyle \frac{dy}{dt} = f(y;p)
$ where $ p $ represents constant parameters. The equilibrium solutions are the
roots of the equations $ f(y) = 0 \implies \displaystyle \frac{dy}{dt} $ which
will depend on the parameters. Bifuraction diagrams are qualitative changes of
the solutions happening when a parameter $ p $ is varied, i.e the change in the
number of stability type of the equilibria. A \textbf{bifuraction diagram}
plots and shows the solutions of branches $ y^*(p) $. 
\subsection{Logistic populations dynamics}
Consider the equation 
$$ \frac{dP}{dt} = \rho \Big (1 - \frac{P}{K} \Big ) P $$
Introduce non-dimensional variables for $ P $ and $ t $ in order to reduce the
number of parameters in the problem. Choose $ P' = \frac{P}{K}  $ and $ t'
= \rho t $. 
$$ \frac{dP'}{dt'} = (1-P')P' $$
The non-dimensional problem does not have any parameters. This shows that we
can't have any bifuractiosn when we modify $ K $, and $ \rho $ as those are all
mathematically equibva;lent problems they can only differ by strecthing or
compeessing the axis $ P $ and $ t $. Now consider the equation 
$$ \frac{dP}{dt} = \rho \Big (1 - \frac{P}{K} \Big) P - F $$
Introduce the same non-dimensional variables for $ P $ and $ t $;
($ P' = \frac{P}{K} $ and $ t' = \rho t $).
$$ \frac{dP'}{dt'} = (1-P')P' - \frac{F}{\rho K} $$
Denoting $ \displaystyle \frac{F}{\rho K} = F' $, $ F' $ cannot be elimated by
using non-dimensional variable, so the problem has 1 real parameter $ \implies
$ it may have qualitatively different solutions when $ F' $ is varied.
\subsection{Harvesting at a constant rate}
$$ \frac{dP'}{dt'} = (1-P')P' - F' $$
How does the equilibria change when $ F' $ is varied? ($P'^*(F') = ? $ ). Solve
it graphically by following the intersections of the two terms on the RHS as
$ F' $ is varied?

\section{Lecture 13 - Systems of Coupled 1st Order ODEs}
$$ y'_1 = f(y_1,y_2), \; y'_2 = g(y_1, y_2) $$
$ y_1(t) = ? $ and $ y_2(t) = ? $. There is no general method for finding
solutions analytically for coupled systems of ODEs (except linear systems), but
often the important information is related to the equilibrium states of the
system. The equilbrium solutions $ (y_1^*, y_2^*) $ are the solutions of the
algebraic system of simultaneous equations:
$$ f(y_1, y_2) = 0 , \; g(y_1, y_2)  = 0 $$
\subsection{Dynamics of interacting populations}
\begin{example}
  Prey $ (P) $ and predator $ (R) $ population dynamics.
  $$ \frac{dP}{dt} = \rho P \Big (1 - \frac{P}{K}\Big) - aPR $$
  $$ \frac{dR}{dt} = bPR - dR $$
  Given $ \displaystyle \rho P\Big ( 1 - \frac{P}{K}\Big ) $ is the prey without predator,
  $ aPR $ is the prey-predator interaction and $ bPR - dR $ is the death of
  predator.
\end{example}
Non-dimensionalize the sytem by introducing new dimensionaless variables $ P'
= \displaystyle \frac{P}{K} $, $ t' = \rho t $ (done similarly for the logistic equation) and
choose $ R' = \displaystyle \frac{a}{\rho} R$. 
$$ \frac{dP'}{dt'} = P'(1-P') - P'R' $$
$$ \frac{dR'}{dt'} = \frac{bK}{\rho} P'R' - \frac{d}{\rho} R' $$
$$ \frac{dR'}{dt'} = \alpha P' R' - \beta R' $$
The non-dimensionalization shows that the behaviour of the solution can only
depend on the two new parameters $ \alpha = b \displaystyle \frac{K}{\rho}
$ and $ \beta = \displaystyle \frac{d}{\rho} $
\section{Lecture 14 - Second Order Differential Equations}
$$ f(y'', y', y,t) = 0, \; y(t) = ?$$
\begin{example}
  Newton's Law $ F = ma $, where $ a $ is the acceleration. (second derivative
  of coordinate function $ a = x''(t)) $
\end{example}
Many partial differentials contain second derivates over spatial coordinates.
There is no general way to solve nonlinear second order ODEs analytically,
however we can solve them numerically.
\par 
To solve numerically, we can rewrite it into the form of two coupled first
order ODEs by introducing an unknown function defined as $ y' = v $. Then $ v'
= F(v,y,t)$ and $ y' = v $ forms a coupled system, needs to be complemented
with initial conditions $ y(t=0)$ and $ v(t=0) = y'(t=0) $
\subsection{Solving linear 2nd order ODEs analytically}
Some types of linear 2nd order ODEs may be solved analytically. A general form
of a analytically solveable 2nd order ODE may be
$$ y'' + p(t) y' + q(t) y = r(x) $$
Can be classified as homogenous if $ r(x) = 0 \; \forall x $. Constant
coefficients if $ p(t)  $ and $ q(t) $ are constants. For homogenous ODEs
($ r(x) = 0 $), we can use the \textbf{Principle of superposition} for
construction the general solution.
\\\\
If $ y_1(t) $ and $ y_2(t) $ are two linearly independent solutions of the
homogenous linear 2nd order ODE then any linear combination $ y(t) = c_1 $.
What?
\subsection{Method of Reduction of Order}
Assume that we have a solution $ y_1(t) $ that satisfies a linear homogenous
ODE $ \implies $ then the reduction of order method leads to a problem of
a lower order ODE for finding the other/general solution. The steps for solving
with reduction of order method is as follows
\begin{enumerate}
  \item $ y'' + p(t) y' + q(t) y = 0 $
  \item Assume that $ y_1 (t) $ is a solution.
  \item Look for solutions of the form $ y = u(t)y_1(t) $
  \item Substitute into the equation.
  \item $ (u y_1)'' + p(t) (u y_1)' + q(t) u y_1 = 0 $
\end{enumerate}
\section{Lecture 15 - Solving 2nd Order ODES}
Suppose we have a linear $ 2^{nd} $ order ODE $ f(y'', y', y, t) = 0 $. Suppose
we are solving for the general solution $ y(t) $. If $ f $ is homogenous and
constant-coefficient, the ODE is of the form $ y'' + ay' + by = 0  $. As
a function that remains the same (same for a multiplicative constant) is $ e^t
$, we can guess $ y(t) = e^{\lambda t} $ is a solution for some $ \lambda $.
\begin{align*}
  y'' + ay' + b &= (e^{\lambda t})'' + a(e^{\lambda t})' + be^{\lambda t}\\
  0 &= \lambda^2 e^\lambda t + \lambda a e^{\lambda t} + be^{\lambda t}\\
  0 &= e^{\lambda t} (\lambda^2 + a\lambda + b)
  \intertext{Note that $ \lambda^2 + a\lambda + b = 0 $ is the characteristic
  equation of an ODE}
\end{align*}
$$ \implies \lambda_{1,2} = \frac{-a \pm \sqrt{a^2 - 4b}}{2} $$
Thus $ e^{\lambda_1 t} $ and $ e^{\lambda_2 t}$ are linearly independent
solutions and the general solution is $ y(t) = c_1 e^{\lambda_1 t} + c_2
e^{\lambda_2 t} $ for some $ c_1, \; c_2 $. However, this is only the case if
$ \lambda_1 \neq \text{something} $ ($ 4b < a^2 $). 
\\\\
If $ 4b = a^2 $, then $ \lambda_1 = \lambda_2 = \lambda $ and we obtain only
one solution $ \lambda^{\lambda t} $. With this, we can use reduction of order.
As $ y_1(t) = e^{\lambda t} $ is a solution, let $ y(t) = u(t) e^{\lambda t} $.
Thus $ y'' + ay' + by = 0 $
$$ \implies (ue^{\lambda t})'' + a(ue^{\lambda t})' + bue^{\lambda t} = 0 $$
\begin{align*}
  (u'e^{\lambda t} + \lambda u e^{\lambda t})' + a(u'e^{\lambda t} + \lambda
  u e^{\lambda t}) + bue^{\lambda t} &= 0\\
  u''e^{\lambda t} + \lambda u' e^{\lambda t} + \lambda u' e^{\lambda t}
  + \lambda^2ue^{\lambda t} + au' e^{\lambda t} + a\lambda u e^{\lambda t}
  + bue^{\lambda t} &= 0\\
  u'' + 2\lambda u' + \lambda^2u + au' + a\lambda u + bu &= 0
\end{align*}
However note that $ \lambda = \displaystyle \frac{-a}{2} $, $ 4b = a^2 \implies
b = \displaystyle \frac{a^2}{4}$
$$ u'' - au' + \frac{a^2}{4}+ au' - \frac{a^2}{2}u + \frac{a^2}{4}u = 0$$
\begin{align*}
  &\implies u'' = 0\\
  &\implies u(t) = c_1t + c_2
\end{align*}
for some $ c_1 $, $ c_2 $. Thus the general solution is
$$ y(t) = u(t) e^{\lambda t} = (c_1 t + c_2) e^{\lambda t} $$
If we obtain no solutions $ (4b > a^2) $ in the real set, we get two complex
conjugate roots.
$$ \lambda_{1,2} = \frac{-a}{2} \pm \frac{\sqrt{4b - a^2}}{2}i = \alpha \pm
\beta i $$
We get the general solution
\begin{align*}
  y(t) &= c_1 e^{(\alpha + \beta i)t} + c_2 e^{(\alpha - \beta i)} \\
       &= c_1 e^{\alpha t} e^{\beta i t} + c_2 e^{\alpha t} e^{-\beta i t}
\end{align*}
However, $ e^{i\beta t} = \cos (\beta t) + i \sin (\beta t) \implies y(t)
= e^{\alpha t} (c_1 \cos(\beta t) + c_1 i \sin (\beta t) + c_2 \cos(-\beta t)
+ i \sin (-\beta t) $
\begin{align*}
  y(t) &= (c_1 + c_2 e^{\alpha t} \cos(\beta t) + (c_1 - c_2) ie^{\alpha t}
  \sin (\beta t)\\
  \intertext{Let $ A = c_1 + c_2 $ and $ B = (c_1 - c_2)i $ }
  y(t) &= Ae^{\alpha t} \cos (\beta t) + B e^{\alpha t} \sin (\beta t)\\
       &= e^{\alpha t} (A\cos (\beta t) + B \sin (\beta t))
\end{align*}
\section{Lecture 16 - Constant-coefficient Non-Homogenous}
Suppose we have an ODE $ y'' + p(t)y' + q(t)y = r(t) $ and we are attempting to
solve for $ y(t) $. This is a constant coefficient non homogenous linear order
$ 2^{nd} $ oder ODE. (if $ p(t) $, $ q(t) $ are constant). Let $ L(x) = x''
+ px' + qx $. Suppose we have two linearly independent solutions $ y_1(t),
y_2(t) $. Thus $ L(y_1) = L(y_2) = r \implies L(y_1) - L(y_2) = 0 $. As the ODE
is linear $ \implies $ $ L $ is linear, $ L(y_1) - L(y_2) = L(y_1 - y_2) = 0 $.
This is a homogenous ODE and can be easily solved
\paragraph{Steps:}
\begin{enumerate}
  \item Solve the corresponding homogenous ODE $ (r=0) \to y_n = c_1 y_n + c_2
    y_k $
  \item Find solution $ y_p $ (reduction of order) $ \to y_p = u(t)y_{k_1} $
  \item General case is $ y(t) = y_p + y_n $
\end{enumerate}
\begin{example}
  Solve for $ y(t) $ given $ y'' + y' - 2y = t^2 - 2t + 3 $
  \begin{enumerate}
    \item Solve $ y'' + y' - 2y = 0 $, characteristic equation is $ \lambda ^2
      + \lambda - 2 = 0$. Implying
      $$ (\lambda+2) (\lambda - 2) = 0 $$
      Thus $ \lambda_1 = -2 $ and $ \lambda _2 = 1 $, therefore
      $$ y_n (t) = c_1 e^{-2t} + c_2 e^t $$
    \item Reduction of Order: as $ y_1(t) = e^t $ is a solution we have
      a linearly independent solution $ y_p(t) = u(t)e^t $. Subbing in 
      \begin{align*}
	(ue^t)'' + (ue^t)' - 2ue^t &= t^2 - 2t + 3\\
	u''e^t + 2u'e^t t + ut^2 e^t + u' e^t + ute^t - 2ue^t &= t^2 - 2t
	+ 3 \\
	u'' + 2u't + ut^2 + u' + u5 - 2u &= (t^2 - 2t + 3)e^{-t}
      \end{align*}
      $$ \implies y(t) = c_1 e^{-2t} + c_2 e^t - \frac{t^2}{2} + \frac{t}{2}
      - \frac{7}{4} $$
  \end{enumerate}
\end{example}

\section{Lecture 17 - }

\section{Lecture 18 - Applications of 2nd order ODEs}
\subsection{Mechanical/Electrical Ossiclators}
\textbf{SMALL ANGLE TIME}
$$ \sin (\theta) = \theta $$
$$ m \frac{dy}{dt} = -mg + kdy  $$
\begin{example}
  Given a pendulum,
  $$ m \frac{d^2y}{dt^2} = - ky $$
  $$ y(0) = y_0, \; y'(0) = 0 $$
\end{example}
\begin{align*}
  my'' + ky &= 0\\
  y'' + \frac{k}{m}y &= 0\\
  \intertext{Denote $ a = \displaystyle \frac{k}{m} $}
  y'' + ay &= 0 \implies \lambda^2 + a = 0 \\
  \therefore y &= e^{\lambda t} 
  \intertext{For $ \lambda = \pm i \sqrt{a} $}
  y(t) &= A \cos (\sqrt{a} t) + B \sin (\sqrt{a} t)
\end{align*}
However with this model there is no dampining over time. 
$$ m \frac{d^2y }{dt^2} = -ky - \gamma y' $$
\begin{align*}
  y'' + \Big ( \frac{\gamma}{m}\Big) y' + \Big ( \frac{k}{m} \Big) y &= 0\\
  \intertext{Denoting $ \displaystyle \frac{\gamma}{m} = b $ and
  $ \displaystyle \frac{m}{k} = a $}
  y'' + by' + ay &= 0
  \intertext{If $ b^2 > 4a $, $ \lambda_1, \lambda_2 < 0 $, if $ b^2 < 4a $,
  $ \lambda_{1,2} = \displaystyle \frac{-b}{2} \pm i\sqrt{4a-b^2} $}
\end{align*}

\section{Lecture 19 - Introduction to Multivariate Calculus}
\subsection{Review of one-variable case}
Let $ f: D \to \mathbb{R} $ be a function with a domain $ D $ an open subset of
$ \mathbb{R} $. For $ a \in D $ we say that the limit $ \displaystyle \lim_{x
\to a}  f(x) $ exists if and only if 
\begin{enumerate}
  \item The limit from the left exists
  \item the limit from the right exists
  \item these two limits coincide etc.
\end{enumerate}
    $$ \lim_{x \to a^-} f(x) = \lim_{x \to a^+}  f(x) $$
    Furthermore if the limit exists and is equal to the actual value of
    $ f $ at $ a $.
    $$ \lim_{x\to a^-}  f(x) = \lim_{x \to a^+}  f(x) = f(a) $$
    We say that $ f $ is continuous at $ x = a $. If $ f $ is continuous on all
    $ D $, we say that $ f $ is continuous function on $ D $
    \subsection{The two-variable case}
    When $ f $ is a function of more than one variable, the situation is more
    subtle. There are more than two ways to approach a given point of interest.
    \begin{example}
    Consider the function
    $$ f(x,y) = \frac{x^2}{x^2 + y^2} $$
    with domain given by $ \mathbb{R}^2 \setminus {(0,0)} $
    \end{example}
    Approaching the origin along $ y = 0 $, if $ x \neq 0  $ $ f(x,0)
    = \displaystyle \frac{x^2}{x^2 + 0} = 1$. Then 
    $$ \lim_{x\to 0} \; f(x,0) = 1 $$
    If $ y \neq 0  $, $ f(0,y) = \displaystyle \frac{0}{0+y^2} = 0 $. Then,
    $$ \lim_{y \to 0}  f(0,y) = 0 $$
    Thus $ \displaystyle \lim_{(x,y) \to (0,0)} f(x,y) $ does not exist, as
    $ \displaystyle \lim_{x\to 0} f(x,0) \neq \displaystyle \lim_{y\to 0}
    f(0,y) $. In general, for the limit $ \displaystyle \lim_{(x,y) \to (a,b)}
    f(x,y)$ to exist, it is neccessary that every parth in $ D $ approaching
    $ (a,b) $ gives the same limiting value ($ (a,b) $ may not neccesarily be
    in $ D $). This gives a method for finding if a limit does not exist for
    multivariate limits.
    \[
      \text{If } \displaystyle \begin{cases}
      f(x,y) \to L_1 \text{ as } (x,y) \to (a,b) \text{ along the path } C_1
      \in D\\
      f(x,y) \to L_2 \text{ as } (x,y) \to (a,b) \text { along the path } C_2
      \in  D
    \end{cases} 
    \begin{remark}
  The above notation is somewhat deficient and perhaps one should write
  $$ \lim_{(x,y) \to (a,b)} f(x,y) $$
  to indicate that only paths $ D $ terminating in $ (a,b) $ (which itself may
  or may not be in $ D $) are considered. For instance,
  $ f(x,y) = x^2 + y^2 $ with $ D = \{(x,y): x^2 + y^2 < 1\} $, then
  $ \displaystyle \lim_{(x,y) \to (1,0)}  f(x,y) $ exists and is 1. However if 
  \[
    \begin{cases}
      x^2 + y^2 &\text{ for } D = \{(x,y): x^2 + y^2 < 1\}\\
      0 &\text{ for } D = \{(x,y): x^2 + y^2 > 1\}
    \end{cases}
    \]
    then $ \displaystyle \lim_{(x,y) \to (1,0)} f(x,y) $ does not exist.
    \end{remark}

    \section{Lecture 20 - Continuation of Functions of Multiple Variables}
    Generally, we write $ \displaystyle \lim_{(x,y) \to (a,b)}  f(x,y) = L $ to
    mean the values of $ f(x,y) $ approach $ L $ as the point $ (x,y)
    $ approaches $ (a,b) $ along any path in the domain $ f $. That is, we can
    make the value of $ f(x,y) $ as close to $ L $ as we like by taking $ (x,y)
    $ sufficiently close to $ (a,b) $. This is formalised through the following
    definition
    \begin{definition}
    Let $ f $ be a function of two variables whose domain $ D $ includes points
    arbitrarily close to $ (a,b) $. 
    \\\\
    Then we say that the limit of $ f(x,y) $ as $ (x,y) $ approaches $ (a,b)
    $ is $ L $ and we write
    $$ \lim_{(x,y) \to (a,b)} f(x,y) = L $$
    if for every number $ \varepsilon > 0 $, $ \exists \delta > 0 $ such that
    if $ (x,y) \in D $ and $ 0 < \sqrt{(x-a)^2 + (y-b)^2} < \delta $, then
    $ |f(x,y) - L| < \varepsilon $. Where $ |f(x,y)-L| $ can be described as
    the distance between $ f(x,y) $ and $ L $ in $ \mathbb{R} $. The
    $ \sqrt{(x-a)^2 + (y-b)^2} $ is the distance between $ (x,y) $ and $ (a,b)
    $ in $ \mathbb{R}^2 $. The definition is essentially saying that the
    distance between $ f(x,y)  $ and $ L $ can be made arbitrarily small by
    making the distance between $ (x,y)  $ and $ (a,b) $ sufficiently small,
    \textbf{but not 0}.
    \end{definition}



    
\end{document}
