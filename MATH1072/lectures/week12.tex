\section{Line Integrals}
\subsection{Work done by a constant force}
In a single dimension, work is done by a constant force \( F \) moving an
object along a straight line of length \( d \) is represented as
\[ 
  W = Fd
\]

In two or three dimensions, work done by a constant force in a moving particle
alng a straight line from \( P \to Q\) is 
\[ 
  W = F \cdot \overrightarrow{PQ} = mgd\cos \theta  
\]
\subsection{Work done over a curve}
We now consider the more general case of the work done by a force field
\[ 
  F(x,y,z) = F_1(x,y,z)i + F_2(x,y,z)j + F_3(x,y,z)k
\]
which moves an object along a curve C.
\\\\
First give an approximation, dividing \( C \) into \( n \) arcs such that the
\textit{i}'th arc has length \( \delta s_i \). We approximate \( \delta s_i \)
by evaluating \( F \) at a specified point \( P_i \) on the arc. 
