% For the CSSE2010 Course

\documentclass{article}
\usepackage[margin=0.7in]{geometry}

\usepackage{circuitikz}

\usepackage{xcolor}
\usepackage{amsmath}
\usepackage{amsthm}
\usepackage{amssymb}

\newtheorem{example}{Example}
\theoremstyle{definition}
\newtheorem{definition}{Definition}

\title{CSSE2010 Notes}
\author{Ismael Khan}

\pagecolor{black}
\color{white}

\usepackage{listings}
\usepackage{color}

\definecolor{dkgreen}{rgb}{0,0.6,0}
\definecolor{gray}{rgb}{0.5,0.5,0.5}
\definecolor{mauve}{rgb}{0.58,0,0.82}

\lstset{frame=tb,
  language=C,
  aboveskip=3mm,
  belowskip=3mm,
  showstringspaces=false,
  columns=flexible,
  basicstyle={\small\ttfamily},
  numbers=none,
  numberstyle=\tiny\color{gray},
  keywordstyle=\color{blue},
  commentstyle=\color{dkgreen},
  stringstyle=\color{mauve},
  breaklines=true,
  breakatwhitespace=true,
  tabsize=3
}

\begin{document}
\section{Lecture 1 - Course Introduction/Bits and Bytes}
\section{Lecture 2- Logic Gates}
\section{Lecture 3 - Binary Arithemetic}
\section{Lecture 4 - Combinational Logic}
\section{Lecture 5 - Flip Flops}
\subsection{Clicker Question}
Consider a multiplexer shown, what must the inpust A,B,C,D be so that the
multiplexer output is 
$$ X = S_1\cdot S_0 + \bar{S_1} \cdot G $$
\subsection{Circuits that remember values}
The output of any logic gate or combinational circuit is dependent on the
inputs. If an input changes, the output can also change. The previous value is
lost forever.

\subsection{Memory Element: D Flip Flop}
\begin{itemize}
  \item D is input
  \item Q is output
  \item CLK (clock) is control input.
\end{itemize}
How does it work? Q copies the value of D (remembers it) whenever CLK goes from
0 to 1 (rising edge).
\subsection{Characteristic Tables}
Characteristic table defines operation of flip-flop in tabular form.a

\subsection{D Flip-flops}
\begin{definition}
  A $ n $ bit \textbf{register} can be made using $ n $ D flip-flops.
\end{definition}
\subsection{Latches from NAND gates}
\begin{circuitikz}[background rectangle/.style={fill=black}, color=white,help lines/.style={color=lightgray,line width=0.2pt}] \draw
(0,2) node[nand port] (1) {}
(0,0) node[nand port] (2) {}

\end{circuitikz}
\subsection{Flip-Flops vs Latches}
Any devices based on edges are referred to as flip flops, these are
edge-triggered devices.\\\\
Triangle indicates edge-triggered (therefore flip-flop)
% Need to figure out how to input the schematics of the circuits

State of the flip flop is the value stored. Flip-flops are more useful than
latches.
\begin{center}
\begin{circuitikz}[background rectangle/.style={fill=black},
  color=white,help lines/.style={color=lightgray,line width=0.2pt}] \draw
(0,2) node[and port] (myand1) {}
(0,0) node[and port] (myand2) {}
(2,1) node[xnor port] (myxnor) {}
(3,1) node[not port] (mynot) {}
(myand1.out) -- (myxnor.in 1)
(myand2.out) -- (myxnor.in 2)
(myxnor.out) -- (mynot.in);
\end{circuitikz}
\end{center}
\section{Lecture 6 - Sequential Circuits (Shift Registers)}

\end{document}


